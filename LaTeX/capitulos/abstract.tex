\begin{resumo}[Abstract]
\begin{otherlanguage*}{english}
The advancements od Cyber-Physical Systems (CPS) has promoted research and development efforts on more complex robotic platforms, demanding modular and scalable architectures. However, the integration between multiple processing units, sensors and actuators still represents a significant challenge in ensuring efficient communication with optimized use of available resources.
This work presents a study on the construction and analysis of architecture in CPSs  using the Robot Operating System 2 (ROS2) and Architecture Analysis Design Language (AADL). Its objective is to evaluate the performance, communication and integration between microcontrollers and distributed processing units in a real robotic application, ensuring the best use of resources in the system.
Model-Driven-Development proved itself capable of foreseeing resource bottlenecks and computational limitations before the physical deployment of the system, ensuring better reliability and efficiency. The results support the joint use of AADL and ROS2 for a more structured, predictable and scalable architecture, favoring the development of systems with a high level of integration, complexity and performance. 
    
\textbf{Keywords}: ROS 2. AADL. Cyber-Physical Systems. Autonomous Systems. Distributed Communication.
\end{otherlanguage*}
\end{resumo}