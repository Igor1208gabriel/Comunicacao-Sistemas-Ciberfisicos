\chapter{Introdução}
A crescente evolução de sistemas ciberfísicos e a demanda para plataformas robóticas autônomas intensificam a demanda de arquiteturas robustas, flexíveis e modulares. Quanto mais sistemas robóticos são empregados em ambientes industriais, comerciais e domésticos, a necessidade de metodologias que garantem confiabilidade, adaptabilidade e escalabilidade aumenta.

Este trabalho aplica conceitos chave de engenharia de software, análise de sistemas e arquitetura no contexto de sistemas robóticos. A modelagem destes demanda uma abordagem multidisciplinar, combinando conhecimento de design de software e hardware, sistemas embarcados, computação distribuída e análise em tempo real.

O tópico foi escolhido devido a importância crescente de sistemas autônomos e a relevância da engenharia baseada em modelos no desenvolvimento de software moderno. A modelagem com a Linguagem de Análise de Arquitetura e Design (AADL) provê uma oportunidade de aplicar conhecimento acadêmico em um contexto altamente prático e relevante industrialmente.

Como apontado por \cite{Mikusz2014}, no campo industrial e empresarial, sistemas robóticos avançados tendem ao aumento da interatividade e cooperação, levando a uma perspectiva onde sistemas Ciberfísicos integram software, hardware e serviços eficientemente. Esta tendência demanda uma abordagem estruturada focada em design de sistemas e validação para garantir confiabilidade, escalabilidade e performance. Em ambientes modernos, é esperado que robôs não apenas executem tarefas pré-programadas, mas também consigam se adaptar a mudanças e desafios inesperados.

O principal objetivo deste trabalho é apresentar um estudo de caso da modelagem arquitetural, com foco na descrição dos processos de desenvolvimento do robô autônomo Robinho, que integra visão computacional, mapeamento de ambiente e algoritmos de reconhecimento de objetos usando o Sistema Operacional Robótico, ROS2, que facilita interações complexas entre unidades de processamento distribuídas, sensores e atuadores.

Para realizar a análise da arquitetura do robô, a AADL é utilizada, tornando possível representar detalhes precisos de hardware e software, além de suas comunicações internas e com o ambiente, garantindo que a integração entre diferentes componentes seja coerente e gargalos na performance possam ser identificados e mitigados antes de uma implantação física.

Os objetivos específicos deste trabalho são:
\begin{itemize}
    \item Descrever a arquitetura de hardware, software e comunicação do sistema robótico Robinho.
    \item Modelar os componentes e suas comunicações dentro do sistema.
    \item Identificar potenciais gargalos ou problemas de integração durante o processo de modelagem.
    \item Medir os benefícios do uso da AADL nos estágios primitivos do desenvolvimento do sistema robótico.
\end{itemize}

A metodologia utilizada preenche a lacuna entre design teórico e aplicação prática, permitindo uma abordagem sistemática para o desenvolvimento de aplicações ciberfísicas confiáveis e eficientes, potencializando o uso de cada componente, físico ou computacional.

O documento é estruturado desta forma: O capítulo 1 introduz a motivação e objetivos do estudo. Capítulo 2 apresenta as fundamentações teóricas de sistemas ciberfísicos, o sistema operacional de robôs, plataformas robóticas e AADL. O Capítulo 3 detalha o estudo de caso envolvendo o robô Robinho, incluindo sua arquitetura e modelagem. O capítulo 4 discute os resultados e desafios encontrados durante o estudo. Finalmente, o capítulo 5 conclui o trabalho e indica sugestões para trabalhos futuros.