\chapter{Fundamentação Teórica}

\section{Sistemas Ciberfísicos}

De acordo com \cite{Moraes2013}, sistemas ciberfísicos integram componentes computacionais com interações físicas, biológicas e de engenharia, incorporando novas capacidades. O custo computacional de um sistema é diretamente proporcional ao número de processos e tarefas simultâneas. Como consequência, um sistema complexo pode chegar rapidamente ao seu limite, causando consequências imprevisíveis se o hardware não for corretamente adequado aos seus casos de uso previstos. \cite{SennBourdon2022}


\section{Modelagem de Sistemas}

O design e validação de sistemas ciberfísicos requer metologias robustas para garantir a integração de componentes de hardware e software mantendo performance, confiabilidade e escalabilidade. Neste estudo, a Architecture Analysis Design Language (AADL) é empregada para modelar a arquitetura de Robinho, provendo uma representação estruturada de seus componentes físicos e computacionais.

AADL permite a definição de componentes de software, como nós de ROS2, elementos de hardware, como microcontroladores e sensores, a interação entre diversos componentes e a implantação de cada pedaço de software em seu respectivo hardware. Esta abordagem garante que todos os aspectos da arquitetura de Robinho sejam bem documentados e analisados antes da implantação, reduzindo gargalos potenciais e melhorando performance e eficiência geral.

Robinho é composto de diversos componentes conectados, cada um mapeado em AADL, com diagramas mostrando claramente suas interações. Isto facilita validação primitiva de escolhas de design. 

Ademais, usando o Open Source AADL Tool Environment, OSATE, a modelagem do sistema é analisada para garantir conformidade com exigências de performance. O processo de modelagem permite identificação prévia de necessidades de recursos, verificação de limites na comunicação e estimativa de carga computacional nas unidades de processamento distribuídas.

Por meio destas ferramentas, a arquitetura de Robinho é otimizada antes de sua implementação física, o que reduz os riscos associados com implantações no mundo real, e como resultado, reduzindo drasticamente os custos gerais.

\section{Arquitetura de Software}
\cite{Bernardo2025, ROSWebsite} O Robot Operating System foi desenvolvido para sustentar uma arquitetura modular e interoperável baseada em pacotes independentes, tornando seu uso adequado para projetos orientados a arquitetura. ROS oferece uma coleção de ferramentas de software para desenvolvimento robótico e testagem, que podem ser reutilizadas facilmente entre diferentes plataformas de hardware.

\section{Sistemas Autônomos}
De acordo com \cite{Wahde2016, Bensalem2009}, sistemas autônomos são sistemas complexos que requerem a interação ou cooperação de inúmeros componentes de software heterogêneos, se movimentando e operando sem a necessidade de supervisão humana direta. É esperado que funcionem em ambientes complexos e não estruturados, e que tomem suas próprias decisões em qualquer situação.

\section{Automatic Code Generation}
OSATE e sua ferramenta experimental RAMSES 2 permitem que desenvolvedores gerem código automaticamente baseado nos modelos e necessidades específicas do sistema.
TODO

\section{Computer Vision Algorithms}
A fazer. Explicar como o sistema utiliza algoritmos de visão computacional, e como o ROS facilita o uso/desempenho.
TODO

\section{Robinho}
Robinho é um sistema robótico (autônomo?) que emprega o ROS2 como framework para comunicação entre seus componentes. O sistema é composto por um ESPCam, um RaspBerryPi e duas rodas atuando como um carrinho de direção diferencial.
A infraestrutura do sistema é melhor descrita em \ref{cap:infra}
TODO