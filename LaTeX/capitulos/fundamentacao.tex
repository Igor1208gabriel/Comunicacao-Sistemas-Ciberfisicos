\chapter{Fundamentação Teórica}
\section{Sistemas Ciberfísicos}
De acordo com \cite{Moraes2013}, sistemas ciberfísicos integram componentes computacionais com interações físicas, biológicas e de engenharia, incorporando novas capacidades. O custo computacional de um sistema é diretamente proporcional ao número de processos e tarefas simultâneas. Como consequência, um sistema complexo pode chegar rapidamente ao seu limite, causando consequências imprevisíveis se o hardware não for corretamente adequado aos seus casos de uso previstos. \cite{SennBourdon2022}
 
\section{Modelagem de Sistemas}

O design e validação de sistemas ciberfísicos requer metologias robustas para farantir a integração de componentes de hardware e software mantendo performance, confiabilidade e escalabilidade. Neste estudo, a Linguagem de Análise de Arquitetura e Design (AADL) é empregada para modelar a arquitetura de Robinho, provendo uma representação estruturada de seus componentes físicos e computacionais
AADL permite a definição de componentes de software, como nós de ROS2, elementos de hardware, como microcontroladores e sensores, a interação entre diversos componentes e a implantação de cada pedaço de software em seu respectivo hardware. Esta abordagem garante que todos os aspectos da arquitetura de Robinho sejam bem documentados e analisados antes da implantação, reduzindo gargalos potenciais e melhorando performance e eficiência geral.
Robinho é composto de diversos componentes conectados, cada um mapeado em AADL, com diagramas mostrando claramente suas interações. Isto facilita validação primitiva de escolhas de design. 
Ademais, usando o Open Source AADL Tool Environment, OSATE, a modelagem do sistema é analisada para garantir conformidade com exigências de performance. O processo de modelagem permite identificação prévia de necessidades de recursos, verificação de limites na comunicação e estimativa de carga computacional nas unidades de processamento distribuídas.

Through this approach, Robinho's architecture is optimized before physical implementation, reducing risks associated with real-world deployment, and as a result, reducing overall costs as well.

\section{Software Architecture}
\cite{Bernardo2025, ROSWebsite} Robot Operating System was designed based on a modular and interoperable architecture based on packages, making them well-suited for architecture-oriented projects. It offers a collection of software tools for robot software development and testing, which can be easily reused across different hardware platforms and integrated onto each other.

\section{Autonomous Systems}
According to \cite{Wahde2016, Bensalem2009}, autonomous robotic systems are complex systems that require the interaction or cooperation of numerous heterogeneous software components, freely moving and operating without direct human supervision. They are expected to function in complex, unstructured environments, and to make their own decisions concerning which action to take in any given situation

\section{Automatic Code Generation}
OSATE and its tool RAMSES 2, developed by Dominique Blouin and his team at the Institut Polytechnique de Paris, allow for developers to generate code automatically based on the system's models and specific necessities.

\section{Computer Vision Algorithms}
TODOTODOTODOTODOTODOTODOTODOTODOTODO