\begin{resumo}
O avanço dos Sistemas Ciberfísicos (CPS) tem impulsionado pesquisa e desenvolvimento de plataformas robóticas cada vez mais complexas, exigindo arquiteturas modulares e escaláveis. Entretanto, a integração entre múltiplas unidades de processamento, sensores e atuadores ainda representa um desafio significativo para garantir comunicação eficiente com uso otimizado dos recursos disponíveis.
Este trabalho apresenta um estudo sobre a construção e análise da modelagem de arquitetura em sistemas ciberfísicos utilizando Robot Operating System 2 (ROS2) e Architecture Analysis Design Language (AADL). O objetivo é avaliar o desempenho, a comunicação e a integração entre microcontroladores e unidades de processamento distribuídas em uma aplicação robótica real, garantindo o melhor uso dos recursos embarcados na aplicação. 
O desenvolvimento dirigido a modelagem se mostrou capaz de prever gargalos de comunicação e limitações computacionais antes mesmo do sistema ser implantado fisicamente, garantindo maior confiabilidade e eficiência. Os resultados obtidos são favoráveis ao uso combinado de AADL e ROS2 para uma arquitetura mais estruturada, previsível e escalável, favorecendo o desenvolvimento de sistemas com alto grau de integração, complexidade e desempenho.

\textbf{Palavras-chave}: ROS 2. AADL. Sistemas Ciberfísicos. Sistemas Autônomos. Comunicação distribuída.
\end{resumo}